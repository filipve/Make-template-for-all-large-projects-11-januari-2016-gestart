\documentclass[a4paper,11pt]{article}
\usepackage{import}
\usepackage{projectstyle}

\usepackage{makeidx}

\usepackage{showidx} %It prints out all index entries in the left margin of the text.

\makeindex

%we have to define a bibliography style in the preamble
\bibliographystyle{plain}

\begin{document}


%\glsaddall
\import{./}{title.tex}

\clearpage

\import{./}{title2.tex}

\clearpage

\thispagestyle{empty}

\tableofcontents

\clearpage



%hoofdstuk 2
\setcounter{section}{1}
\section{Applicatie laag}
\import{hoofdstuk2sections/}{deel2-1.tex}

%hoofdstuk 7
%\setcounter{section}{6}
%\newpage
%\section{Multimedia netwerken}
%\import{hoofdstuk7sections/}{deel7-1.tex}

%toont source code als algoritme
\begin{algorithm}[ht]
%dit zorgt ervoor dat het in je list of algoritms wordt getoond
\caption{See how easy it is to provide algorithms}
\label{myFirstAlgorithm}
\begin{algorithmic}
\REQUIRE $a$
\STATE $b = 0$
\STATE $x \leftarrow 1:10$
\FORALL{x}
    \STATE $b = b+a$
\ENDFOR
\RETURN $b$
\end{algorithmic}
\end{algorithm}


%Next we are adding the additional lists to the end of the document:

\listofalgorithms
\clearpage
\listoffigures
\clearpage
\listoftables
\clearpage

%\printglossaries
\printglossary[title=Termen,toctitle=Lijst van termen]

\printglossary[type=\acronymtype]


\clearpage
%\import{./}{bibliography.tex}

\end{document}
